\renewcommand{\vec}[1]{\boldsymbol{\mathbf{#1}}}
% BEGIN COMMANDS FROM LATEX FOR PHYSICISTS
% http://www.dfcd.net/articles/latex/latex.html
\newcommand{\ket}[1]{| #1 \rangle} % for Dirac bras
\newcommand{\bra}[1]{\langle #1 |} % for Dirac kets
\newcommand{\bket}[1]{\right| #1 \right\rangle} % for Dirac bras
\newcommand{\bbra}[1]{\left\langle #1 \left|} % for Dirac kets

\newcommand{\ccr}[1]{c^{\dagger}_{#1}}
\newcommand{\can}[1]{c_{#1}}

\newcommand{\com}[2]{\left[#1, #2\right]}
\newcommand{\acom}[2]{\left\{#1, #2\right\}}


% Added a braket command that takes either two or three arguments, so that \braket{a}{b} = <a|b> and \braket{a}{b}{c} =
% <a|b|c>
\newcount\seccount

\def\braket{%
    \seccount0%
    \let\go\secnext\go
}

\def\secnext#1{%
    \def\last{\left\langle #1}%
    \futurelet\next\secparse
}

\def\secparse{%
    \ifx\next\bgroup
                  \let\go\secparseii
    \else
        \let\go\seclast
    \fi
    \go
}

\def\secparseii#1{%
    \ifnum\seccount>0| \fi
    \advance\seccount1\relax
    \last
    \def\last{#1}%
    \futurelet\next\secparse
}

\def\seclast{ \ifnum\seccount>0{} | \fi\last \right\rangle}%
% END COMMANDS FROM
